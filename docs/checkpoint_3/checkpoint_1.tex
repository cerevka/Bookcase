\documentclass[12pt]{article}

\usepackage[utf8]{inputenc}
\usepackage[czech]{babel}
\usepackage{a4wide}
\usepackage{graphicx}
\usepackage{url}
\linespread{1.3}

\usepackage{fancyhdr}
\pagestyle{fancy}
\fancyhf{}

\renewcommand{\headrulewidth}{0.4pt}
\renewcommand{\footrulewidth}{0.4pt}

% Nastaveni hlavicky a paticky.
\lhead{X33EJA - Enterprise Java}
\rhead{Tomáš Čerevka, Adam Činčura}
\lfoot{\today}
\rfoot{\thepage}

\begin{document}

\noindent
\begin{Huge}Bookcase - checkpoint 3\end{Huge}

\section{Popis aplikace}

Cílem aplikace Bookcase bylo vytvořit rozhraní, které umožní uživatelům evidovat své osobní knihovničky, sdílet je s vybranými přáteli a hlavně udržovat přehledně informace o tom, kdo má který svazek půjčen a kdy by ho měl vrátit.

Aplikace sice není realizována v plném rozsahu, nicméně její hlavní funkcionalita je zprovozněna a zbytek by se realizoval velmi podobnými postupy. Konkrétně v aplikaci funguje:

\begin{itemize}
	\item Registrace uživatelů.
	\item Přihlašování a odhlašování uživatelů.
	\item Posílání e-mailů.
	\item Vložení nové knihy do databáze:
		\begin{itemize}
			\item s již existujícím autorem.
			\item s vytvořením nového autora.
		\end{itemize}
	\item Procházení knih, které uživatel vlastní.
	\item Procházení vše knih v databázi.
	\item Umístění knihy do košíku.
	\item Odebrání knihy z košíku.
	\item Zapůjčení všech knih v košíku.
	\item Celé rozhraní aplikace je multijazyčné.
\end{itemize} 

\section{Architektura a technologie}

Pro vývoj jsme používali aplikační server Glassfish 3.1, Netbeans 7.0 a MySQL 5.5.12 na operačním systému Fedora 15 a Windows.

Aplikace se skládá z jednoho EJB modulu, který obsahuje stateless a stateful (košík) session beans, jednu message driven bean, která zajišťuje odesílání e-mailů přes SMTPS gmailu, dále z webového modulu, který využivá hlavně technologie JSF 2.0 a při stylování pomáhal CSS framework Blueprint CSS. Oba dva tyto moduly jsou zastřešeny pod jednou enterprise aplikací.

\section{Zhodnocení}

Osobně jsem v Enteprise Javě EE 6 pracoval již dříve (na své bakalářské práci), nicméně tam se jednalo o navazující projekt. Díky této semestrální práci jsem si vyzkoušel vytvoření projektu na "zelené louce".

Velmi se mi líbí práce s objektově-relačním mapováním a zpracováním logiky. Naopak se mi nelíbí způsob navigace mezi stránkami a volání akcí přes commandLink či commandButton, které využívají ke své práci JavaScript. Navíc url adresy se netváří příliš "seo-friendly". Doufám, že tento nedostatek dokáže vyřešit doplněk pro JSF jménem PrettyFaces, který bych rád vyzkoušel.

Na druhou stranu mi tato technologie zatím připadá stále poněkud těžkopádná. Měli jsme nemalé problémy zprovoznit Glassfish na dvou strojích i přesto, že jsme ho zcela identicky nakonfigurovali a nasadili na něj stejný EAR. Těžko říci, jestli to bylo rozdíly mezi Linuxovou a Windows verzí.

\section{Instalace}

Projekt je hostován na \url{https://github.com/cerevka/Bookcase}, kde jsou k dizpozici zdrojové kody a instrukce k nastavení projektu (\url{https://github.com/cerevka/Bookcase/wiki/Bookcase}).

Projekt lze také nainstalovant přiloženým skriptem pro Shell, který je schopný čistě nainstalovaný Glassfish nastavit od A do Z a nadeployovat na něj přiložený EAR. Pouze je třeba připravit v databázi view pro logování a uživatelské skupiny.

\end{document}
